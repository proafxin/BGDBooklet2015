\documentclass{subfile}

\begin{document}
	\section{Geometry}
	
	\begin{problem}
		A point $P$ is chosen in the interior of $\triangle ABC$ so that when lines are drawn through $P$ parallel to the sides of $\triangle ABC$, the resulting smaller triangles $t_{1},t_{2},t_{3}$ in $\triangle ABC$ have areas $4,9$ and $49$ respectively. Find the area of $\triangle ABC$.
	\end{problem}
	
	\begin{solution}
		Let the line through $P$ parallel to $BC$ intersect $AB,AC$ at $D,E$ respectively. Again, let the line through $P$ parallel to $CA$ intersect $BC,AB$ at $F,G$ respectively. Finally, let the line through $P$ parallel to $AB$ intersect $BC,CA$ at $K,L$ respectively. Assume that $\triangle PKF=t_1, \triangle PEL=t_2, \triangle PDG=t_3$. 
		
		Now, $\triangle PKF\sim \triangle LPE\sim \triangle GDP\sim \triangle ABC$, and $AGPL, BDPK, CEPF$ are all parallellograms. Next, $\dfrac{EC}{LE}=\dfrac{PF}{LE}=\sqrt{\dfrac{(KPF)}{(PLE)}}=\sqrt{\dfrac{4}{9}}=\dfrac{2}{3}$. 
		Similarly, $\dfrac{AL}{LE}=\dfrac{7}{3}$. So, $\dfrac{AC}{LE}=\dfrac{AL+LE+EC}{LE}=\dfrac{AL}{LE}+\dfrac{LE}{LE}+\dfrac{EC}{LE}=4$. So $\dfrac{(ABC)}{(LPE)}=(\dfrac{AC}{LE})^2=16\Rightarrow (ABC)=144$. 
	\end{solution}
	
	\newpage
	\begin{problem}
		A convex hexagon $ABCDEF$ is inscribed in a circle such that $AB=CD=EF$ and diagonals $AD,BE$ and $CF$ are concurrent. Let $P$ be the intersection of $AD$ and $CE$. prove that,
		\[\dfrac{CP}{PE}=(\dfrac{AC}{CE})^2\]
	\end{problem}
	
	\begin{solution}
		Let $Q$ be the concurrency point of the diagonals $Ad,BE,CF$. Now, note that according to a very well-known lemma, $\dfrac{CP}{PE}=\dfrac{CA\cdot \sin \angle CAD}{BF\cdot \sin \angle DAE}$. Next, since $AB=EF$, $ABEF$ must be an isosceles trapezoid, which means $AE=BF$. Similarly, $DF=CE$. Now, $\dfrac{CE}{BF}=\dfrac{CQ}{DQ}=\dfrac{CQ}{DQ}\cdot \dfrac{DQ}{BQ}=\dfrac{CQ}{DQ}\cdot \dfrac{DE}{AB}=\dfrac{CQ}{DQ}\cdot \dfrac{DE}{CD}=\dfrac{CA}{DF}\cdot \dfrac{\sin \angle CAD}{\sin \angle DAE}=\dfrac{CA}{CE}\cdot \dfrac{\sin \angle CAD}{\sin \angle DAE}$. 
		
		Now, $\dfrac{CP}{PE}=\dfrac{CA\cdot \sin \angle CAD}{BF\cdot \sin \angle DAE}=\dfrac{CA}{CE}\cdot \dfrac{CE}{BF}\cdot \dfrac{\sin \angle CAD}{\sin \angle DAE}=(\dfrac{CA}{CE})^{2}$ from the previous relations.
	\end{solution}
	
	
	\begin{problem}
		Let $ABCD$ be a convex quadrilateral such that diagonals $AC$ and $BD$ intersect at right angles, and let $E$ be their intersection. Prove that the reflections of $E$ across $AB,BC,CD,DA$ are concyclic. 
	\end{problem}
	
	
	\begin{solution}
		Let the reflections of $E$ across $AB,BC,CD,DE$ be $P,Q,R,S$ respectively. Now, $AP=AE=AS$, i.e., $A$ is the circumcenter of $\triangle PSE$. So, $\angle SPE=\frac{1}{2}\angle SAE=\angle DAE$. Similarly, $\angle EPQ=\angle EBC, \angle ERQ=\angle ECB, \angle ERS=\angle EDA$. So, $\angle SPQ+\angle SRQ=\angle SPE+\angle EPQ+\angle ERQ+\angle ERS=\angle DAE+\angle EBC+\angle ECB+\angle EDA=180^{\circ}-\angle AED+180^{\circ}-\angle BEC=180^{\circ}$, since $\angle AED=\angle BEC=90^{\circ}$. So $PQRS$ is cyclic.
	\end{solution}
	
	\newpage
	\begin{problem}
		Let $O$ be the circumcenter of a triangle $\triangle ABC$ and let $\ell$ be the line going through the midpoint of the side $BC$ and which is perpendicular to the bisector of $\angle BAC$. Find the value of $\angle BAC$ if the line $\ell$ goes through the midpoint of the line segment $AO$.
	\end{problem}
	
	\begin{solution}
		There are two parts in this solution, actually. The first part is to prove that $\angle BAC$ is obtuse. The second part is using this information to get the correct figure and evaluate the desired angle. 
		
		For the first part, note that unless $\angle BAC$ is obtuse, the line $\ell$ can't intersect the segment $AO$. 
		
		For the second part, let $M,L$ be the midpoints of $BC,AO$ respectively. Then $ML$ is the line $\ell$. Again, let $A'$ be the midpoint of arc $BC$ that does not contain $A$. Then $AA'$ is the bisector of $\angle BAC$. Let $N$ be the mispoint of $AA'$. And let $ML$ intersect $AA'$ at $K$. So, $MK\perp AA',ON\perp AA'$.
		
		Now, clearly $L$ is the center of $\odot AON$. So, $LA=LN$. But $LK\perp AN$. So $AK=KN$. This means $KL\parallel ON\Rightarrow LM\parallel NO$. Again, $\angle LNA=\angle LAN=\angle OAA'=\angle OA'A\Rightarrow LN\parallel MO$. So $LMON$ is a parallelllogram. Now, $OM=NL=LA=\frac{1}{2}OA=\frac{1}{2}OC$.
		
		Now, in $\triangle OCM$, $\angle OMC=90^{\circ}$ and $OM=\frac{1}{2}OC$. From these, it is an easy drill to prove that $\angle OCM=30^{\circ}$. A little angle chase from there yields $\angle BAC=120^{\circ}$. 
	\end{solution}
	
	\newpage
	\begin{problem}
		\textbf{An old IMO problem:} A triangle $\triangle A_{1}A_{2}A_{3}$ and a point $P_{0}$ are given in the plane. We define
		
		\begin{center} $A_{s}=A_{s-3} \forall s\geq 4$ \end{center}
		
		We construct a sequence of points $P_{1},P_{2},...$ such that $P_{k+1}$ is the image of $P_{k}$ under rotation with center $A_{k+1}$ through an angle $120$ degree clockwise (for $k=0,1,2,...$).
		
		Prove that if $P_{1986}=P_{0}$, then the triangle $\triangle A_{1}A_{2}A_{3}$ is equilateral. 
	\end{problem}
	

		
		\newpage
		
		\section{Number Theory}
		
		\begin{problem}[Masum Billal]
			An integer is called square-free if it doesn't have any divisor that is a perfect square greater than $1$. Prove that $a^{a-1}-1$ is never square-free for $a>2$.
		\end{problem}
		
		\begin{solution}[\bf First]
			\textit{Lifting the Exponent Lemma} totally kills this problem.
			
			\begin{lemma}[Lifting the Exponent Lemma(LTE)]
				If $p$ is a prime divisor of $x-y$ where $\gcd(x,y)=1$, then \[\nu_p(x^n-y^n)=\nu_p(x-y)+\nu_p(n)\]
			\end{lemma} 
			
			Assume $p$ is a prime divisor of $a-1$. Then, by the lemma,
			\begin{eqnarray*}
				\nu_p(a^{a-1}-1) & = & \nu_p(a-1)+\nu_p(a-1)\\
				& = & 2\nu_p(a-1)\\
				&\geq&2
			\end{eqnarray*}
			Therefore, $p^2|a^{a-1}-1$ and it's not square-free.
		\end{solution}
		
		\begin{solution}[\bf Second]
			This is a better solution that uses nothing.
			\[a^{a-1}-1=(a-1)(a^{a-2}+\ldots+a+1)\]
			Let $m=a-1$. Then, $a\equiv1\pmod m$ and
			\begin{eqnarray*}
				a^{a-2}+\ldots+a+1  &\equiv& 1^{a-2}+\ldots+1+1\pmod m\\
				&\equiv&m\equiv0\pmod m
			\end{eqnarray*}
			Therefore, $a^{a-1}-1$ is divisible by $m^2$.
		\end{solution}
		
		\begin{note}
			The second solution also provides a stronger claim.
		\end{note}
		
		\begin{problem}
			Determine if $2^{2015}+3^{2015}+4^{2015}+5^{2015}$ is a prime.
		\end{problem}
		
		\begin{solution}
			Well, this was a problem so everyone solves at least two(paired with problem \eqref{e2}). No solution provided for this one.
		\end{solution}
		
		\begin{problem}
			For a prime $p>3$, prove that $\binom{2p-1}{p-1}-1$ is divisible by $p^3$.
		\end{problem}
		
		\begin{solution}
			\begin{theorem}[Wolstenholme's Theorem]
				For a prime $p>3$, \[\binom{ap}{bp}\equiv\binom{a}{b}\pmod {p^3}\]
			\end{theorem}
			Set $a=2,b=1$. We have, \[\binom{2p}{2}\equiv\binom{2}{1}\equiv2\pmod {p^3}\]
			Remember that, $\binom{n}{k}=\dfrac{n}{k}\binom{n-1}{k-1}$, so \[\binom{2p}{p}=2\binom{2p-1}{p-1}\]
			Therefore, $p^3$ divides $2\binom{2p-1}{p-1}-2=2\left(\binom{2p-1}{p-1}-1\right)$. Since $(p^3,2)=1$, we can say $p^3$ divides $\binom{2p-1}{p-1}-1$.
		\end{solution}
		
		\begin{problem}\label{e2}
			For integers $a,b$, prove that $a^pb-ab^p$ is divisible by $p$.
		\end{problem}
		
		\begin{problem}
			\begin{theorem}[Fermat's Little Theorem]
				For any prime $p$ and an integer $a$, $p$ divides $a^p-a$. Particularly, if $p$ doesn't divide $a$ i.e. $(a,p)=1$, 
				\[a^{p-1}\equiv1\pmod p\]
			\end{theorem}
			Write $a^pb-ab^p=ab(a^{p-1}-b^{p-1})$. If one of $a$ or $b$ is divisible by $p$, we are done. If neither of them is divisible by $p$,
			\[a^{p-1}\equiv1\equiv b^{p-1}\pmod p\]
			Thus, $p$ divides $a^{p-1}-b^{p-1}$.
		\end{problem}
		
		\begin{problem}[Masum Billal]
			Find the number of positive integers $d$ so that $d$ divides $a^n-a$ for all integer $a$ where $n$ is a fixed natural number.
		\end{problem}
		
		\begin{solution}
			Let's assume $n>1$. 
			
			\begin{lemma}
				$d$ is square-free.
			\end{lemma}
			
			\begin{proof}
				Let $p$ be a prime so that $p^2$ divides $d$. Then setting $a=p$, we get $p^2|p^n-p$ or $p^2|p$, which is a contradiction. Thus, no square of a prime divides $d$ i.e. $d$ is square-free.
			\end{proof}
			
			\begin{lemma}
				If $n$ has $k$ distinct prime factors, it has at least $2^k$ divisors.
			\end{lemma}
			
			\begin{proof}
				Let $n=\prod\limits_{i=1}^{k}p_i^{e_i}$. Then since $e_i\geq1$,
				\begin{eqnarray*}
					\tau(n) & = & \prod\limits_{i=1}^{k}(e_i+1)\\
					&\geq& \prod_{i=1}^{k}2\\
					& = & 2^k
				\end{eqnarray*}
			\end{proof}
			
			\begin{theorem}
				For a prime $p$, there are $\t (p)$ primitive roots. In particular, a prime $p$ has a primitive root.
			\end{theorem}
			
			\begin{theorem}\label{ord}
				If $h=\ord_n(a)$ and $n$ divides $a^k-1$, then $h$ divides $k$.
			\end{theorem}
			
			\begin{lemma}
				$p-1$ divides $n-1$.
			\end{lemma}
			
			\begin{proof}
				Without loss of generality, $p$ must divide $a^{n-1}-1$ for integer $(a,p)=1$. Since we are free to choose $a$, we choose a primitive root $g$ of $p$. Then $p$ divides $g^{n-1}-1$ and $p$ divides $g^{p-1}-1$. Because $\ord_p(g)=p-1$, we have by theorem \eqref{ord} that $p-1$ divides $n-1$.
			\end{proof}
			
		\end{solution}
		
		\begin{problem}[Masum Billal]
			For a positive real number $c>0$, call a positive integer $n$, $c-good$ if for all positive integer $m<n$, $\dfrac mn$ can be written as \[\dfrac mn=\dfrac{a_0}{b_0}+\ldots+\dfrac{a_k}{b_k}\]
			for some non-negative integers $k,a_0,...,a_k,b_0,...,b_k$ with $k<\dfrac nc, 2b_k< n$ and $0\leq a_i< \min(b_j),0\leq j<k$.
			Show that, for any positive real $c$ there are infinite $c-good$ numbers.
		\end{problem}
		
		\begin{solution}
			Consider a prime $p\geq3$. Then any number can be written in $p$-base as
			\[m=a_kp^k+\ldots+a_1p+a_0\]
			where $0\leq a_i\leq p-1$Therefore, if $n=p^r$ with $r>k$, \[\dfrac{m}{n}=\dfrac{a_k}{p^{r-k}}+\ldots+\dfrac{a_1}{p^{r-1}}+\dfrac{a_0}{p^r}\]
			$a_i<p\leq \min(b_j)=p^{r-k}$, $2p^{r-k}<p^r$ and $k\leq \log_p{m}<\log_p{n}<\dfrac{n}{c}$ since $n$ can be arbitrary large but $c$ is fixed. Fixing $c$, since we can choose any odd prime, we have infinite such $c$-good number.
		\end{solution}
		
		\begin{problem}[Masum Billal]
			Define $\Lambda(n)=\log p$ if $n=p^m$ for some positive integer $m$, otherwise $\Lambda(n)=0$ and $G(n)=\sum\limits_{i=1}^{n} \Lambda(i)$. Prove that for $n>10$, $\L(n)+\log(n+1)\geq n\log2$.
		\end{problem}
		
		\begin{hint}
			This is actually a very weak version of an inequality I found. Here is a hint on where you should start. Take logarithm in both sides of \[\binom{2n}{n}=\prod\limits_{p\leq2n}\sum\limits_{i=1}^{\lfloor\log_p{2n}\rfloor}\left\lfloor\dfrac{2n}{p^i}\right\rfloor-2\left\lfloor\dfrac{n}{p^i}\right\rfloor\]
			Also, use the fact that $\left\lfloor\dfrac{2n}{p^i}\right\rfloor-2\left\lfloor\dfrac{n}{p^i}\right\rfloor\in\{0,1\}$.
		\end{hint}
		
		\newpage
		
		\section{Combinatorics}
		
		\begin{problem}
			In a picnic, let there be $1^2$ student from Class One, $2^2$ students from Class Two, $3^2$ students from Class Three, $4^2$ students from Class Four and $5^2$ students from Class Five. A teacher is picking students for a game at random. How many students must he pick to make sure that there are at least $10$ students from the same class?
		\end{problem}
		
		\begin{problem}
			In a party, there are $n$ people and their shoes are in $n$ lockers. After the party, electricity went out and everyone forgot the number of locker his/her shoe was in. So they take the shoes randomly. What's the probability that all of them got their own shoes?
		\end{problem}
		
		\begin{problem}
			You have $n$ jewels, but exactly one of them is a fake. You know that the fake jewel is lighter. With a scale balance, how many measurements are sufficient to find the fake jewel?
		\end{problem}
		
		\begin{solution}
			The answer is $\boxed{\log_3{n}}$. Here is an algorithm:
		\end{solution}
		
		\begin{problem}
			There are $n$ ants on a $p$ meter rope, on which each walks on a $v$ $m/s$ speed. It is known that $(1)$ when two ants collide on the rope, they turn around and continue to move the way they came from at the same speed, and $(2)$ when an ant reaches the end of a rope they fall off from it. Find the greatest amount of time after which every single ant must fall off the rope, and find
			the arrangement for which that is possible.
		\end{problem}
		
		\begin{problem}
			We have $2015$ points in the plane such that any three are not collinear. Prove that there is a circle which contains $1007$ points in its interior and another $1007$ points in its exterior.
		\end{problem}
		
		\begin{solution}
			Let's say we have already found the circle and it has center $O$ and radius $R$. $1007$ points are strictly outside and $1007$ are inside, this means the other point must be on the boundary. This is quite useful, which tells us to consider the distances of the points from the center. Call the points $P_1,...P_{n}$ where $n=2015$. Without loss of generality, we can assume that $P_1008$ lies on the boundary and the points $OP_1,...,OP_{1007}$ are inside the circle of radius $OP_{1008}$. Then $OP_{1009},...,OP_{2015}$ are outside the circle. If $OP_i$ is inside the circle then we must have $OP_i<OP_{1008}$, otherwise $OP_i>OP_{1008}$. This should tell you to sort the distances somehow. In other words, we need a construction for the center $O$ so that the distances of $P_i$ are sorted. We are done if we can find $O$ so that all the distances are distinct. In order to find such a construction, we can think the opposite. When will two distances be equal? $OP_i=OP_j$ is possible only if $O$ lies on the perpendicular bisector of $P_iP_j$. Since we want all the distances distinct, we need to take $O$ so that it doesn't lie on any perpendicular bisector of $P_iP_j$ for all $i,j$. And obviously there are infinite such points. Now, we can sort the points according to distances i.e. $OP_1<OP_2<...<OP_{2015}$. Therefore, we make $O$ center and draw a circle with radius $OP_{1008}$ and we are done.
		\end{solution}
		
		\begin{problem}
			Can you choose $1983$ pairwise distinct integers each less than $100000$ such that no three are in an arithmetic progression?
		\end{problem}
		
		\begin{problem}
			Show that for $n>2$, there is a set of $2^{n-1}$ points in the plane, no three collinear such that no $2$n form a convex $2n$-gon.
		\end{problem}
		
		\newpage
		
		\section{Mock Exam $1$}
		
		\newpage
		
		\section{Mock Exam $2$}
		
		\begin{problem}
			Determine all triples of positive integers $(k,m,n)$ so that $2^k+3^m+1=6^n$.
		\end{problem}
		
		\begin{problem}
			Let $\Gamma$ be the circumcircle of a triangle $\Delta ABC$. Let $\ell$ be a line tangent to $\Gamma$ at point $A$.Let $D, E$ be interior points of the sides $AB, AC$ respectively, which satisfy the condition $\dfrac{BD}{DA}=\dfrac{AE}{EC}$. Let $F, G$ be the two points of intersection of line $DE$ and circle $\Gamma$. Let $H$ be the	point of intersection of the line $\ell$ and the line parallel to $AC$ and going through point $D$. Let $I$ be the point of intersection of the line $\ell$ and the line parallel to $AB$ and going through $E$.
			Prove that the four points $F, G, H, I$ lie on the circumference of a circle which is tangent to line $BC$.
		\end{problem}
		
		\begin{problem}
			Let $n$ be a positive integer. For every pair of students enrolled in a certain school having $n$ students, either the pair are mutual friends or not mutual friends. Let N be the smallest possible sum, a + b, of positive integers a and b satisfying the following two conditions concerning students in this school.
			\begin{enumerate}
				\item It is possible to divide students into a teams in such a way that any pair of students belonging to the same team are mutual friends
				\item It is possible to divide students into b teams in such a a way that any pair of students belonging to the same team are not mutual friends.
			\end{enumerate}
			Assume that every student will belong to one and only one team when the students are divided into teams that satisfy the conditions above. A team may consist of only one student, in which case this team is assumed to satisfy both of the conditions: that any pair of students in this team are mutual friends; are not mutual friends. Determine in terms of $n$ the maximum
			possible value that $N$ can take.
		\end{problem}
\end{document}